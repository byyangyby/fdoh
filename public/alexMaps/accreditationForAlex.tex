\documentclass{article}
\usepackage{geometry}
\usepackage{graphicx}
\renewcommand{\familydefault}{\sfdefault}
\usepackage{helvet}
\title{Communicable Disease Reporting}
\author{Florida Department of Health - Alachua County - Disease Control Unit}
\usepackage{Sweave}
\begin{document}
\Sconcordance{concordance:accreditationForAlex.tex:accreditationForAlex.Rnw:%
1 7 1 1 0 4 1 1 8 34 1}

\maketitle
\begin{center}
\end{center}

\section*{Summary}
The Alachua County Health Department reports primary and secondary data to the State in multiple formats.

\section*{Reportable diseases}
As per  statutes (Florida Adminstrative Code, 64D-3), the Alachua County Disease Control Unit reports all cases of reportable diseases/conditions of which they are notified to the Bureau of Epidemiology. The steps for reporting are as follows: \\
\begin{enumerate}
\item A health practitioner or laboratory identifies a case of a reportable disease/condition.
\item Following legal guidelines, this practitioner or laboratory contacts the County Epidemiologist to inform her of the case.
\item The County Epidemiologist requests the necessary data for reporting to the State (this varies widely depending on the disease in question).
\item The County Epidemiologist counducts follow-up (patient interview, request for further laboratory evidence, etc.).
\item The County Epidemiologist provides a full report to the State via Merlin (the online reporting platform).\\
\end{enumerate}

Appendix 1 displays the total number of diseases/conditions reported by the ACHD to the FDOH Bureau of Epdemiology during the period from 12/30/2012 to 12/09/2013.


\section*{Non-reportable diseases: the example of influenza}
The collection of data on "reportable" diseases constitutes one part of how ACHD reports to the State.  There are also "non-reportable" conditions (ie, practitioners and laboratories are not obligated to report to ACHD) on which the ACHD Disease Control Unit compiles secondary data for the purpose of informing the State. \\

The best example of this "secondary" data reporting for non-reportable conditions is the case of influenza.  Influenza is a condition of very serious public health significance, but is not "reportable."  Accordingly, the ACHD employs two measures of active surveillance as a means to gauging influenza activity in the area.\\  
\begin{enumerate}
\item Emergency Department (ED) data is compiled by local EDs and sent to the ACHD daily.  An ACHD analyst subsets the data for ILI (influenza-like illness) and reports trends daily to the County Epidemiologist.  
\item School nurses at all 41 public schools monitor students for ILI and report this data to an ACHD analyst.  This analyst compiles the data, examines it for geographical clusters and trends and reports to the County Epidemiologist.\\  
\end{enumerate}
Based on these two data sources, the County Epidemiologist is able to report weekly local ILI activity (both in terms of directional trending as well as in relation to the baseline) to the State.  These reports are made every Tuesday via EpiGateway, an online reporting portal through which FDOH's Bureau of Epidemiology monitors influenza activity in all counties.\\


\section*{Appendices}
\begin{enumerate}
\item Diseases/conditions reported to the State, 12/30/2012- 12/09/2013
\item Example of the daily ED surveillance report with ILI details
\item Example of the weekly school ILI surveillance report
\end{enumerate}
\end{document}
